%===============================================================================
% МИНИМАЛИСТИЧЕН ATS-ПРИЯТЕЛСКИ CV ШАБЛОН
% Автор: Михаил Доброславски
% Лиценз: MIT
%===============================================================================

\documentclass[11pt, a4paper]{article}

%-------------------------------------------------------------------------------
% ПАКЕТИ
%-------------------------------------------------------------------------------
\usepackage[utf8]{inputenc}
\usepackage[T1]{fontenc}
\usepackage[T2A]{fontenc}
\usepackage[bulgarian]{babel}
\usepackage[margin=0.45in]{geometry}
\usepackage{hyperref}
\usepackage{titlesec}
\usepackage{enumitem}
\usepackage{xcolor}

%-------------------------------------------------------------------------------
% КОНФИГУРАЦИЯ
%-------------------------------------------------------------------------------
% Акцентен цвят (тъмносин, смени на черен за чисто черно-бяла версия)
\definecolor{accent}{RGB}{0, 51, 102}

% Стил на хипервръзките
\hypersetup{
    colorlinks=true,
    linkcolor=accent,
    urlcolor=accent,
    pdfauthor={Михаил Доброславски},
    pdftitle={Михаил Доброславски - CV}
}

% Добавяне на подчертаване за всички хипервръзки
\let\oldhref\href
\renewcommand{\href}[2]{\oldhref{#1}{\underline{#2}}}

% Премахване на номера на страници
\pagestyle{empty}

% Форматиране на секции
\titleformat{\section}
    {\large\bfseries\color{accent}}
    {}
    {0em}
    {}
    [\titlerule]

\titlespacing{\section}{0pt}{8pt}{4pt}

% Настройки на списъци
\setlist[itemize]{
    leftmargin=1.5em,
    itemsep=0pt,
    parsep=0pt,
    topsep=1pt
}

%-------------------------------------------------------------------------------
% ПЕРСОНАЛИЗИРАНИ КОМАНДИ
%-------------------------------------------------------------------------------

% Команда за заглавна част
% Използване: \cvheader{Име}{Email}{GitHub}{LinkedIn}
\newcommand{\cvheader}[4]{
    \begin{center}
        {\LARGE\bfseries #1}\\[4pt]
        \href{mailto:#2}{#2} \quad | \quad
        \href{#3}{GitHub} \quad | \quad
        \href{#4}{LinkedIn}
    \end{center}
    \vspace{-6pt}
}

% Команда за профил
% Използване: \cvprofile{Текст}
\newcommand{\cvprofile}[1]{
    \section{Профил}
    #1
}

% Команда за опит/проект
% Използване: \cventry{Заглавие}{Организация}{Дата}{Локация}{Технологии}
\newcommand{\cventry}[5]{
    \noindent\textbf{#1} \hfill \textit{#3}\\
    \textit{#2} \hfill #4\\
    \textbf{Технологии:} #5
    \vspace{1pt}
}

% Команда за образование
% Използване: \cveducation{Степен}{Институция}{Дата}{Детайли (опционално)}
\newcommand{\cveducation}[4]{
    \noindent\textbf{#1} \hfill \textit{#3}\\
    \textit{#2}\\
    #4
}

% Команда за умения
% Използване: \cvskill{Категория}{Умения}
\newcommand{\cvskill}[2]{
    \noindent\textbf{#1:} #2
}

%===============================================================================
% ДОКУМЕНТ
%===============================================================================
\begin{document}

%-------------------------------------------------------------------------------
% ЗАГЛАВНА ЧАСТ
%-------------------------------------------------------------------------------
\cvheader
    {Михаил Доброславски}
    {mihail.dobroslavski@gmail.com}
    {https://github.com/MishoMish/}
    {https://www.linkedin.com/in/mihail-dobroslavski-4826b831a/}

%-------------------------------------------------------------------------------
% ПРОФИЛ
%-------------------------------------------------------------------------------
\cvprofile{
    Студент по Компютърни науки с 2 години производствен опит в медийния сектор. Силен фокус върху фронтенд разработка с React и TypeScript, с познания в бекенд услуги и облачно внедряване. Асистент и член на студентски съвет с експертиза в системно програмиране. Контрибутор към проекти с отворен код и лични проекти в GitHub профила.
}

%-------------------------------------------------------------------------------
% УМЕНИЯ
%-------------------------------------------------------------------------------
\section{Умения}

\cvskill{Езици}{C++, TypeScript, JavaScript, SQL, HTML/CSS, Java}\\[1pt]
\cvskill{Frameworks \& Libraries}{React, Node.js, Koa, Tailwind CSS, video.js, Knex}\\[1pt]
\cvskill{Инструменти и платформи}{PostgreSQL, Docker, Azure, Git, Keycloak, Storybook}\\[1pt]
\cvskill{Концепции}{REST APIs, Автентикация (JWT/OAuth), CI/CD, Infrastructure as Code, Контейнеризация}

%-------------------------------------------------------------------------------
% ОПИТ
%-------------------------------------------------------------------------------
\section{Опит}

\cventry
    {Софтуерен инженер}
    {Българска национална телевизия}
    {Януари 2024 -- Декември 2025}
    {София, България}
    {React, TypeScript, Node.js, Koa, PostgreSQL, Keycloak, Azure, Docker, Terraform}

\begin{itemize}
    \item Участие в разработката на \textbf{BNT Play}, национална VOD и стрийминг платформа, обслужваща \textbf{20 хиляди активни потребители} в инфраструктурата на българското национално радиотелевизионно излъчване.
    \item Разработка на преизползваема UI компонентна библиотека с \textbf{Storybook} и TypeScript, стандартизираща дизайн системата на BNT Play платформата.
    \item Имплементация на видео възпроизвеждане чрез \textbf{video.js}, поддържащо \textbf{HLS стрийминг} за живи предавания и прогресивно MP4 доставяне за видео при заявка.
    \item Сътрудничество в малък екип за разработка на производствени функционалности, използвайки установена Azure инфраструктура и CI/CD работни процеси.
\end{itemize}

\vspace{2pt}

\cventry
    {Асистент}
    {Факултет по математика и информатика, Софийски университет}
    {Септември 2023 -- Настояще}
    {София, България}
    {C++, Структури от данни и програмиране, ООП}

\begin{itemize}
    \item Менторство на \textbf{75+ студенти} в 3 основни курса по компютърни науки, провеждане на седмични лабораторни сесии и код ревюта.
    \item Разработка на \textbf{автоматизирани тестови случаи} за оценяване на задачи, подобряване на ефективността и консистентността на оценяването.
    \item Предоставяне на индивидуално менторство по сложни теми като управление на паметта и ООП принципи, помагайки на студентите да овладеят фундаментални концепции.
\end{itemize}

%-------------------------------------------------------------------------------
% ОБРАЗОВАНИЕ
%-------------------------------------------------------------------------------
\section{Образование}

\cveducation
    {Бакалавър по Компютърни науки}
    {Софийски университет „Св. Климент Охридски", Факултет по математика и информатика}
    {2022 -- 2026 (Очаквана)}
    {Релевантни курсове: Структури от данни, Алгоритми, Операционни системи, Бази данни, ООП, Компютърни мрежи}

%-------------------------------------------------------------------------------
% ЛИДЕРСТВО И ОБЩНОСТ
%-------------------------------------------------------------------------------
\section{Лидерство и общност}

\noindent\textbf{Член на студентски съвет} \hfill \textit{Април 2024 -- Настояще}\\
\textit{Факултет по математика и информатика}
\begin{itemize}
    \item Организиране на \textbf{FMI Codes}, един от най-големите студентски хакатони в България с \textbf{100+ участници} в 18 отбора; ръководене на техническата продукция на \href{https://www.youtube.com/watch?v=UXKOYHhelAc}{7-часова стрийм церемония по закриване}.
    \item Координиране на логистика и адаптация на 300+ първокурсници по време на \textbf{Freshers Weekend FMI Edition}.
    \item Ръководене на организационния екип за \textbf{FMI Game Olympics} турнири по шах, настолни игри, тенис на маса и електронни спортове.
\end{itemize}

%-------------------------------------------------------------------------------
% ЕЗИЦИ
%-------------------------------------------------------------------------------
\section{Езици}

\noindent Български (Роден) \quad | \quad Английски (B2--C1, Професионално работно владеене)

\end{document}
